% Options for packages loaded elsewhere
\PassOptionsToPackage{unicode}{hyperref}
\PassOptionsToPackage{hyphens}{url}
%
\documentclass[
]{article}
\usepackage{lmodern}
\usepackage{amssymb,amsmath}
\usepackage{ifxetex,ifluatex}
\ifnum 0\ifxetex 1\fi\ifluatex 1\fi=0 % if pdftex
  \usepackage[T1]{fontenc}
  \usepackage[utf8]{inputenc}
  \usepackage{textcomp} % provide euro and other symbols
\else % if luatex or xetex
  \usepackage{unicode-math}
  \defaultfontfeatures{Scale=MatchLowercase}
  \defaultfontfeatures[\rmfamily]{Ligatures=TeX,Scale=1}
\fi
% Use upquote if available, for straight quotes in verbatim environments
\IfFileExists{upquote.sty}{\usepackage{upquote}}{}
\IfFileExists{microtype.sty}{% use microtype if available
  \usepackage[]{microtype}
  \UseMicrotypeSet[protrusion]{basicmath} % disable protrusion for tt fonts
}{}
\makeatletter
\@ifundefined{KOMAClassName}{% if non-KOMA class
  \IfFileExists{parskip.sty}{%
    \usepackage{parskip}
  }{% else
    \setlength{\parindent}{0pt}
    \setlength{\parskip}{6pt plus 2pt minus 1pt}}
}{% if KOMA class
  \KOMAoptions{parskip=half}}
\makeatother
\usepackage{xcolor}
\IfFileExists{xurl.sty}{\usepackage{xurl}}{} % add URL line breaks if available
\IfFileExists{bookmark.sty}{\usepackage{bookmark}}{\usepackage{hyperref}}
\hypersetup{
  pdftitle={Homework1},
  pdfauthor={Jeff Nguyen},
  hidelinks,
  pdfcreator={LaTeX via pandoc}}
\urlstyle{same} % disable monospaced font for URLs
\usepackage[margin=1in]{geometry}
\usepackage{graphicx,grffile}
\makeatletter
\def\maxwidth{\ifdim\Gin@nat@width>\linewidth\linewidth\else\Gin@nat@width\fi}
\def\maxheight{\ifdim\Gin@nat@height>\textheight\textheight\else\Gin@nat@height\fi}
\makeatother
% Scale images if necessary, so that they will not overflow the page
% margins by default, and it is still possible to overwrite the defaults
% using explicit options in \includegraphics[width, height, ...]{}
\setkeys{Gin}{width=\maxwidth,height=\maxheight,keepaspectratio}
% Set default figure placement to htbp
\makeatletter
\def\fps@figure{htbp}
\makeatother
\setlength{\emergencystretch}{3em} % prevent overfull lines
\providecommand{\tightlist}{%
  \setlength{\itemsep}{0pt}\setlength{\parskip}{0pt}}
\setcounter{secnumdepth}{-\maxdimen} % remove section numbering

\title{Homework1}
\author{Jeff Nguyen}
\date{31/10/2020}

\begin{document}
\maketitle

\textbf{Pre-MFE Probability Seminar}\\
\textbf{Baruch College, Fall 2020}

Homework 01\\
Student Name: Ngoc Son (Jeff) Nguyen

\hypertarget{question-1}{%
\section{\texorpdfstring{\textbf{Question
1}}{Question 1}}\label{question-1}}

The sample space \(\Omega\) is designed as follow:\\
\hspace*{0.333em} 1. It is a set of tuples of 3, each tuple represent an
outcome.\\
\hspace*{0.333em} 2. Each tuples contains 3 ordered elements, each
elements represent a box in the order of 1, 2 and 3.\\
\hspace*{0.333em} 3. Either of the tuple elements can be an empty list
{[}{]}, representing an empty box.\\
\hspace*{0.333em} 4. Otherwise, each tuple element draws from the list
\(L = [1, 2, 3, 4, 5]\) representing the numbered balls.\\
\hspace*{0.333em} 5. Beside {[}{]}, each tuple element is an ordered
list of number from \(L\) with the total number of list elements ranging
from 1 to 5, representing 1 to 5 ~~ ball(s) in each box.\\
\hspace*{0.333em} 6. The outcome set is a combination of: (0,0,5),
(0,1,4), (0,2,3), (0,3,2), (0,4,1), (1,1,4), (1,2,3), (1,3,2), (1,4,1),
(2,1,2), (2,2,1), (3,1,1), each ~~ tuple representing a way to put 5
balls in each box.\\
\hspace*{0.333em} 7. Note that the order of the balls inside each
box--tuple element--does not matter,
i.e.~({[}{]},{[}{]},{[}1,2,3,4,5{]}) is the same as
({[}{]},{[}{]},{[}2,1,4,3,5{]}).

Thus, we have the sample space as follow:\\
\begin{equation}
  \begin{aligned}
    \Omega = \{{([], [], [1,2,3,4,5]), ([], [1,2,3,4,5], []), ([1,2,3,4,5], [], []), \\
              ([], [1], [2,3,4,5]), ([], [2], [1,3,4,5]), ([], [3], [1,2,4,5]), ([], [4], [1,2,3,5]), ([], [5], [1,2,3,4]), \\
              ([1], [], [2,3,4,5]), ([2], [], [1,3,4,5]), ([3], [], [1,2,4,5]), ([4], [], [1,2,3,5]), ([5]. [], [1,2,3,4]), \\
              ([1], [2,3,4,5], []), ([2], [1,3,4,5], []), ([3], [1,2,4,5], []), ([4], [1,2,3,5], []), ([5], [1,2,3,4], []),  \\
              ([], [1,2], [3,4,5]), ([], [1,3], [2,4,5]), ([], [1,4], [2,3,5]), ([], [1,5], [2,3,4]), ([], [2,3], [1,4,5]), \\ 
              ([], [2,4], [1,3,5]), ([],[2,5], [1,3,4]), ([], [3,4], [1,2,5]), ([], [3,5], [1,2,4]), ([], [4,5], [1,2,3]), \\
              ([1,2], [], [3,4,5]), ([1,3], [], [2,4,5]), ([1,4], [], [2,3,5]), ([1,5], [], [2,3,4]), ([2,3], [], [1,4,5]), \\
              ([2,4], [], [1,3,5]), ([2,5], [], [1,3,4]), ([3,4], [], [1,2,5]), ([3,5], [], [1,2,4]), ([4,5], [], [1,2,3]), \\
              ([1,2], [3,4,5], []), ([1,3], [2,4,5], []), ([1,4], [2,3,5], []), ([1,5]), [2,3,4], []), ([2,3], [1,4,5], []), \\
              ([2,4], [1,3,5], []), ([2,5], [1,3,4], []), ([3,4], [1,2,5], []), ([3,5], [1,2,4], []), ([4,5], [1,2,3], []), \\
              ([], [1,2,3], [4,5]), ([], [1,2,4], [3,5]), ([], [1,2,5], [3,4]), ([], [2,3,4], [1,5]), ([], [2,3,5], [1,4]), ([], [3,4,5], [1,2])  \\
              ([1,2,3], [], [4,5]), ([1,2,4], [], [3,5]), ([1,2,5], [], [3,4]), ([2,3,4], [], [1,5]), ([2,3,5], [], [1,4]), ([3,4,5], [], [1,2])  \\
              ([1,2,3], [4,5], []), ([1,2,4], [3,5], []), ([1,2,5], [3,4], []), ([2,3,4], [1,5], []), ([2,3,5], [1,4], []), ([3,4,5], [1,2], []) \\
              ([], [1,2,3,4], [5]), ([], [2,3,4,5], [1]), ([], [1,3,4,5], [2]), ([], [1,2,4,5], [3]), ([], [1,2,3,5], [4]), \\
              ([1,2,3,4], [], [5]), ([2,3,4,5], [], [1]), ([1,3,4,5], [], [2]), ([1,2,4,5], [], [3]), ([1,2,3,5], [], [4]), \\
              ([1,2,3,4], [5], []), ([2,3,4,5], [1], []), ([1,3,4,5], [2], []), ([1,2,4,5], [3], []), ([1,2,3,5], [4], []), \\
              ([1], [2], [3,4,5]), ([1], [3], [2,4,5]), ([1], [4], [2,3,5]), ([1], [5], [2,3,4]), \\
              ... \\
              ([1], [2,3], [4,5]), ([1], [2,4], [3,5]), ([1], [2,5], [3,4]) \\
              ... \\
              ([1,2], [3], [4,5]), ([1,2], [4], [3,5]), ([1,2], [5], [3,4])  \\
              ... }\}
  \end{aligned}
\end{equation}

Based on combinatorial counting methods, we know the number of sequences
of length \(5\) with \(3\) symbols is \(n^k = 3^5 = 243\). Thus there
are \(243\) outcomes in the set \(\Omega\).

\hypertarget{a-box-i-remains-empty}{%
\subsection{(a) box I remains empty}\label{a-box-i-remains-empty}}

The subset of outcomes where box I remains empty is:\\
\begin{equation}
  \begin{aligned}
    A_1 = \{{([], [], [1,2,3,4,5]), ([], [1,2,3,4,5], []), \\
              ([], [1], [2,3,4,5]), ([], [2], [1,3,4,5]), ([], [3], [1,2,4,5]), ([], [4], [1,2,3,5]), ([], [5], [1,2,3,4]), \\
              ([], [1,2], [3,4,5]), ([], [1,3], [2,4,5]), ([], [1,4], [2,3,5]), ([], [1,5], [2,3,4]), ([], [2,3], [1,4,5]), \\ 
              ([], [2,4], [1,3,5]), ([],[2,5], [1,3,4]), ([], [3,4], [1,2,5]), ([], [3,5], [1,2,4]), ([], [4,5], [1,2,3]), \\
              ([], [1,2,3], [4,5]), ([], [1,2,4], [3,5]), ([], [1,2,5], [3,4]), ([], [2,3,4], [1,5]), ([], [2,3,5], [1,4]), ([], [3,4,5], [1,2])  \\
              ([], [1,2,3,4], [5]), ([], [2,3,4,5], [1]), ([], [1,3,4,5], [2]), ([], [1,2,4,5], [3]), ([], [1,2,3,5], [4]), }\}
  \end{aligned}
\end{equation}

We can count 28 outcomes in \(A_1\), thus
\(P(A_1) = \frac{\lvert A_1 \rvert}{\lvert \Omega \rvert} = \frac{28}{243} \approx .1152263374 \approx 11.5226 \%\)

\hypertarget{b-at-most-one-box-remains-empty}{%
\subsection{(b) at most one box remains
empty}\label{b-at-most-one-box-remains-empty}}

We are looking for the subset of outcomes where at most one box remains
empty, i.e.~all outcomes but ones where two boxes are empty.\\
We have the outcomes where two boxes are empty: \begin{equation}
  \begin{aligned}
    \Omega \backslash A_2 = \{{([], [], [1,2,3,4,5]), ([], [1,2,3,4,5], []), ([1,2,3,4,5], [], [])} \}
  \end{aligned}
\end{equation}

We can count 3 outcomes in \(\Omega \backslash A_2\), thus
\(P(A_2) = \frac{\lvert \Omega \rvert - \lvert \Omega \backslash A_2 \rvert}{\lvert \Omega \rvert} = \frac{243-3}{243} \approx ..98765432 \approx 98.7654 \%\)

\hypertarget{a-box-i-and-ii-remains-empty}{%
\subsection{(a) box I and II remains
empty}\label{a-box-i-and-ii-remains-empty}}

The subset of outcomes where box I remains empty is:\\
\begin{equation}
  \begin{aligned}
    A_3 = \{{([], [], [1,2,3,4,5])} \}
  \end{aligned}
\end{equation}

We can count 1 outcomes in \(A_3\), thus
\(P(A_3) = \frac{\lvert A_3 \rvert}{\lvert \Omega \rvert} = \frac{1}{243} \approx 4.115226337e^{-3} \approx 0.4115 \%\)

\hypertarget{question-2}{%
\section{\texorpdfstring{\textbf{Question
2}}{Question 2}}\label{question-2}}

\end{document}
